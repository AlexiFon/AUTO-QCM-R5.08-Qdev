\documentclass{article}

    \usepackage[latin1]{inputenc}
    \usepackage[T1]{fontenc}

    \usepackage[bloc,completemulti]{automultiplechoice}
    \usepackage{multicol}
    \begin{document}

    \AMCrandomseed{1237893}

    \element{amc}{
      \begin{question}{Question1}
    \bareme{b=1}
    Texte
    \begin{multicols}{2}
      \begin{reponses}
        \bonne{Réponse}
      \end{reponses}
    \end{multicols}
  \end{question}
}

    \exemplaire{10}{

    %%% début de l'en-tête des copies :

    \noindent{\bf Classe d'application d'AMC  \hfill Examen du 01/01/2010}

    \vspace{2ex}

    Cet examen a été réalisé avec l'application AUTO-QCM

    \vspace{3ex}

    \noindent\AMCcode{etu}{8}\hspace*{\fill}
    \begin{minipage}{.5\linewidth}
    $\longleftarrow{}$ codez votre numéro d'étudiant ci-contre, et écrivez votre nom et prénom ci-dessous.

    \vspace{3ex}

    \champnom{\fbox{
        \begin{minipage}{.9\linewidth}
        Nom et prénom :

        \vspace*{.5cm}\dotfill
        \vspace*{1mm}
        \end{minipage}
    }}\end{minipage}

    \vspace{1ex}

    \noindent\hrulefill

    \vspace{2ex}

    \begin{center}
    Les questions faisant apparaître le symbole \multiSymbole{} peuvent présenter zéro, une ou plusieurs bonnes réponses. Les autres ont une unique bonne réponse.
    \end{center}

    \noindent\hrulefill

    \vspace{2ex}
    %%% fin de l'en-tête

    \melangegroupe{amc}
    \restituegroupe{amc}

    \clearpage

    }

    \end{document}
    